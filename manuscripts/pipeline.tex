\documentclass{article}\usepackage[]{graphicx}\usepackage[]{color}
% maxwidth is the original width if it is less than linewidth
% otherwise use linewidth (to make sure the graphics do not exceed the margin)
\makeatletter
\def\maxwidth{ %
  \ifdim\Gin@nat@width>\linewidth
    \linewidth
  \else
    \Gin@nat@width
  \fi
}
\makeatother

\definecolor{fgcolor}{rgb}{0.345, 0.345, 0.345}
\newcommand{\hlnum}[1]{\textcolor[rgb]{0.686,0.059,0.569}{#1}}%
\newcommand{\hlstr}[1]{\textcolor[rgb]{0.192,0.494,0.8}{#1}}%
\newcommand{\hlcom}[1]{\textcolor[rgb]{0.678,0.584,0.686}{\textit{#1}}}%
\newcommand{\hlopt}[1]{\textcolor[rgb]{0,0,0}{#1}}%
\newcommand{\hlstd}[1]{\textcolor[rgb]{0.345,0.345,0.345}{#1}}%
\newcommand{\hlkwa}[1]{\textcolor[rgb]{0.161,0.373,0.58}{\textbf{#1}}}%
\newcommand{\hlkwb}[1]{\textcolor[rgb]{0.69,0.353,0.396}{#1}}%
\newcommand{\hlkwc}[1]{\textcolor[rgb]{0.333,0.667,0.333}{#1}}%
\newcommand{\hlkwd}[1]{\textcolor[rgb]{0.737,0.353,0.396}{\textbf{#1}}}%
\let\hlipl\hlkwb

\usepackage{framed}
\makeatletter
\newenvironment{kframe}{%
 \def\at@end@of@kframe{}%
 \ifinner\ifhmode%
  \def\at@end@of@kframe{\end{minipage}}%
  \begin{minipage}{\columnwidth}%
 \fi\fi%
 \def\FrameCommand##1{\hskip\@totalleftmargin \hskip-\fboxsep
 \colorbox{shadecolor}{##1}\hskip-\fboxsep
     % There is no \\@totalrightmargin, so:
     \hskip-\linewidth \hskip-\@totalleftmargin \hskip\columnwidth}%
 \MakeFramed {\advance\hsize-\width
   \@totalleftmargin\z@ \linewidth\hsize
   \@setminipage}}%
 {\par\unskip\endMakeFramed%
 \at@end@of@kframe}
\makeatother

\definecolor{shadecolor}{rgb}{.97, .97, .97}
\definecolor{messagecolor}{rgb}{0, 0, 0}
\definecolor{warningcolor}{rgb}{1, 0, 1}
\definecolor{errorcolor}{rgb}{1, 0, 0}
\newenvironment{knitrout}{}{} % an empty environment to be redefined in TeX

\usepackage{alltt}

\usepackage{authblk}

\title{Reproducible workflows and collaboration with domain-knowledge experts}
\author[1]{Charles T. Gray}
\author[2]{Hollie Birkinshaw}
\author[3]{Matthew Grainger}
\author[2]{Tamar Pincus}
\author[1]{Gavin Stewart}
\affil[1]{Newcastle University}
\affil[2]{Royal Holloway}
\affil[3]{NINA}

% code macros
% code
\newcommand{\code}[1]{\texttt{\detokenize{#1}}}
% package
\newcommand{\package}[1]{\texttt{\detokenize{#1::}}}
% function
\newcommand{\function}[1]{\texttt{\detokenize{::#1}}}
% package::function
\newcommand{\pkgfn}[2]{\texttt{\detokenize{#1::#2}}}

\usepackage{tabularx}
\IfFileExists{upquote.sty}{\usepackage{upquote}}{}
\begin{document}

\maketitle


\begin{knitrout}
\definecolor{shadecolor}{rgb}{0.969, 0.969, 0.969}\color{fgcolor}\begin{kframe}
\begin{alltt}
\hlkwd{library}\hlstd{(tidyverse)}
\hlkwd{library}\hlstd{(targets)}
\hlkwd{library}\hlstd{(gt)}
\hlkwd{library}\hlstd{(xtable)}

\hlstd{knitr}\hlopt{::}\hlstd{opts_chunk}\hlopt{$}\hlkwd{set}\hlstd{(}\hlkwc{echo} \hlstd{=} \hlnum{TRUE}\hlstd{)}

\hlstd{knitr}\hlopt{::}\hlstd{opts_knit}\hlopt{$}\hlkwd{set}\hlstd{(}\hlkwc{root.dir} \hlstd{= here}\hlopt{::}\hlkwd{here}\hlstd{())}
\end{alltt}
\end{kframe}
\end{knitrout}


\begin{knitrout}
\definecolor{shadecolor}{rgb}{0.969, 0.969, 0.969}\color{fgcolor}\begin{kframe}
\begin{alltt}
\hlkwd{source}\hlstd{(}\hlstr{"R/hpp_themes.R"}\hlstd{)}
\end{alltt}
\end{kframe}
\end{knitrout}


\section{Unboxing the black box of data wrangling}

% fraud < mistakes
% mistakes caused by lack of domain knowledge
% transparency _during_ the research process

Proponents of open science commendably underscore reuse and extensibility of 
scientific research components, such as data and code; techniques that facilitate the incorporation of these components into future analyses~\cite{laine_2018, peng_reproducible_2011-1, wilkinson_2016}. Less explicit attention, however, is given to the benefits of reproducible workflows, where results can be readily calculated by another researcher, and computational transparency \emph{during} the research project, as opposed to beyond publication. Of course, we can appeal to an analyst's commitment to scientific civic duty, but by explicitly examining how reproducible workflows can facilitate collaboration with domain-knowledge experts we begin to answer the arguably more pertinent question, \emph{What's in reproducibility for me}?      

\subsection{Benefits of reproducibility and open science}

Much has been said about the benefits of reproducibility, wherein precisely the same computational results are readily produced by another analyst, and open science. We differentiate between explicit replication of results, where the same scientific conclusion is drawn from a repeated experiment, as was the focus of the  semantically confusing Reproducibility Project~\cite{collaboration_2015}.  However, it is worth emphasising that reproducibility in and of itself does not necessarily further the `process of scientific discovery'~\cite{devezer_2019}. And in the context of open science, preregistratation, where a clear outline of a scientific methodology is published prior to embarking on the work, does not protect against poor conclusions~\cite{szollosi_2019b}. These discussions are central and ongoing in the meta-research community; a weighing of merits of pathways to scientific discovery, how to find truths or draw conclusions.

In this manuscript, we do not seek to contribute to these worthy discussions. Instead, our focus is on scientific computational \emph{practice}, as opposed to conclusion. Setting aside which path leads us best to a result, let us instead focus on an aspect of computational reproducibility that provides utility for the analyst in avoiding pitfalls and foibles of answering scientific questions. For science has necessarily become ever more collaborative, as the demand for more complex algorithms rises. 

In the example discussed in this manuscript, we examine the implementation of a network meta-analysis in a Cochrane review of the treatment of chronic pain with antidepressants~\cite{birkinshaw_2021}. A project such as this is necessarily collaborative; it is unrealistic to expect a single academic to provide expert-level data science, statistical analysis, and domain-knowledge insights. Scientific fraud is, of course, something we seek to guard against, but here we are more concerned with garden-variety errors the data-analyst and statistician, who by nature are discipline agnostic, make in absence of a genuine understanding of the domain. In so doing, we hope to contribute to the growing literature on selfish reasons for an analyst to adopt computational reprudicbility~\cite{markowetz_2015}.

\subsection{The problem of black box analysis: pitfalls, foibles, and outright fraud}

Reproducibility can certainly guard against fraud. By reproducing analyses, scientists have uncovered cases of outright manipulation of data and results~\cite{baggerly_2009, schleunes_2020}. Reproducibility provides a transparency that assists in uncovering academic misconduct of this kind. However, there is also a grey area of \textbf{Questionable Research Practices}, such as choosing variables to maximise statistical significance, wherein a scientist diligently follows methods they were trained in but lead to spurious results~\cite{fraser_questionable_2018}. Here we will focus not on questionable practices in statistical modelling, but in the preparation of data for modelling, variously known as \textbf{cleaning} or \textbf{wrangling}.

Much focus is given to correct model design. However, there is a deceptively mundane requirement before modelling can begin. Commonly, variables of interest must be encoded with finite levels in columns in a flat-form delimited data format, such as \code{.csv}. For this Cochrane review, we wish to build network meta-analyses, but in addition, we wish to create summary of findings tables, as well as provide the other scientists with an accessibly formatted delimited dataset that they can readily open in whatever spreadsheet software they are comfortable with. 


\begin{kframe}
\begin{alltt}
\hlkwd{tar_read}\hlstd{(p_output_raw)} \hlopt \hlkwd{xtable}\hlstd{()} \hlopt
  \hlkwd{print}\hlstd{(}\hlkwc{tabular.environment} \hlstd{=} \hlstr{"tabularx"}\hlstd{,}
        \hlkwc{width} \hlstd{=} \hlstr{"\textbackslash{}\textbackslash{}textwidth"}\hlstd{)}
\end{alltt}
\end{kframe}% latex table generated in R 4.1.0 by xtable 1.8-4 package
% Wed Aug  4 17:15:54 2021
\begin{table}[ht]
\centering
\begin{tabularx}{\textwidth}{rllllllllll}
  \hline
 & outcome & study & arm & obs & obs\_info & type & design & condition & class & main\_aim \\ 
  \hline
1 & mood, pain, etc. & unique study identifier & unique arm identifier & columns with mean, sd, counts, sample size, etc. & information extracted from column headers:
        timepoint, scale, etc. & Intervention Type & Group & Chronic pain conditions(s) & Intervention Class & Main aim (Pain... \\ 
   \hline
\end{tabularx}
\end{table}



\subsection{A reproducible workflow for collaboration}

From here we will detail an R-specific workflow using the \package{targets} package~\cite{landau_2021}, Google sheets, with data-scraping provided by the \package{googlesheets4} package~\cite{bryan_2021}.

\section{Labelling scales and interpreting dosage}

\section{Discussion}

\bibliographystyle{plain}
\bibliography{biblio}

\end{document}
